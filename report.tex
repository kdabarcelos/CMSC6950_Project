\documentclass{article}
\usepackage{graphicx}
\usepackage{float}

\title{CMSC6950 Project Report: Tidynamics}
\author{Karina Barcelos}
\date{Spring - June 2021}

\setlength{\oddsidemargin}{0.5cm}
\setlength{\textwidth}{15.5cm}
\setlength{\topmargin}{-1.5cm}
\setlength{\textheight}{22cm}

\begin{document}
\maketitle

\section{Introduction}

The aim of this project was to implement two computational tasks from a chosen open source package, such as tidynamics\cite{Buyl2018}. For the first task, there is available bond length of C-H and C=O over 700 ns of a Molecular Dynamics (MD) simulation from a certain Tyrosine amino acid. For the second task, the script generated random values for the random walk coordinates. Several modules were utilized during the project, including, numpy, pandas, math, sys, and matplotlib.  


\section{Results}

Text

\subsection{Task 1}

Text 

\begin{figure}[H]
\includegraphics[width=\linewidth]{CO_CH_length_acf_plot.png}
\caption{XXC-H and C=O bond length over Molecular Dynamics time and their autocorrelation function.}
\label{acf_plot}
\end{figure}




\subsection{Task 2}


Text

\begin{figure}[H]
\includegraphics[width=\linewidth]{msd_plot.png}
\caption{2D random walk in a (x,y) coordinates and its mean-squared displacement.}
\label{msd_plot}
\end{figure}



\section{Conclusions}


\bibliography{refs}

\end{document}


\bibliographystyle{plain}
\bibliography{references}


\end{document}
